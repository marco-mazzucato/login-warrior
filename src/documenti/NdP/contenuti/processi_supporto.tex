\chapter{Processi di Supporto}

\section{Documentazione}
\section{Gestione della Configurazione}
\section{Gestione della Qualità}
\section{Verifica}

\subsection{Scopo}
La verifica ha come obiettivo il controllo che un prodotto sia corretto e completo.

\subsection{Descrizione}
Nella verifica si prende in input ogni processo e lo si controlla, in modo da identificare dubbi o incorrettezze.
Si applica la verifica su un processo quando:
\begin{itemize}
  \item Si raggiunge un livello di maturità adeguato e sufficiente.
  \item A seguire di un cambiamento di stato.
\end{itemize}

\subsection{Aspettative}
Per assicurare il corretto svolgimento della verifica si devono rispettare i seguenti punti:
\begin{itemize}
  \item Seguire procedure definite.
  \item Seguire criteri validi e affidabili.
  \item Ogni prodotto deve passare attraverso fasi successive, che verranno quindi verificate.
\end{itemize}
Una volta terminata la fase di verifica, rispettando i punti appena citati si può quindi proseguire alla fase di \textbf{Validazione}.

\subsection{Verifica della documentazione}
Il processo di verifica per quanto riguarda la documentazione può essere riassunto in queste attività:
\begin{itemize}
  \item Controllo dell'ortografia e della sintassi.
  \item Controllo della stesura del documento.
  \item Controllo della pertinenza e della correttezza dei contenuti del documento.
\end{itemize}

\subsubsection{Attività di analisi statica}
Questa tipologia di analisi risulta molto utile per la verifica di un documento. Si divide in:
\begin{itemize}
  \item \textbf{Walkthrough}: Tecnica dove il verificatore va alla ricerca di eventuali errori attraverso una lettura ad ampio spettro.
  \item \textbf{Inspection}: Tecnica dove il verificatore va alla ricerca di specifici errori attraverso letture mirate.
\end{itemize}
Inizialmente l'attività di Walkthrough sarà prevalente rispetto a Inspection ma
con l'avanzamento dell'attività di progetto e quindi con il ripetersi del processo di Verifica sulla documentazione, si riuscirà a sviluppare una lista degli errori più comuni detta \textbf{Lista di Controllo} consentendo l'utilizzo più massivo della tecnica \textbf{Inspection} che risulta essere più efficiente.

\section{Validazione}
