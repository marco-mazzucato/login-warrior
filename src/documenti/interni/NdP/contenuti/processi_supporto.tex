\chapter{Processi di Supporto}

\section{Documentazione}
\section{Gestione della Configurazione}
\subsection{Scopo}
    Lo scopo di questa sezione è quello di espilicitare come il gruppo intende gestire la produzione di 
    risorse durante l'intero sviluppo del progetto.
 \subsection{Aspettative}
    Le attese che il gruppo MERL si pone sono:
    \begin{itemize}
        \item Semplificare l'individuazione di conflitti ed errori
        \item Uniformare gli strumenti utilizzati
        \item Tracciare le modifiche e avere sempre a disposizione una versione precedente per ogni file
    \end{itemize}
 \subsection{Descrizione}
    Il processo di gestione della configurazione ha lo scopo di creare un ambiente in cui la produzione di 
    documentazione e di codice avviene in maniera sistematica, ordinata e standardizzata. Ciò è permesso 
    raggruppando ed organizzando tutti gli strumenti e le regole adoperati.  
 \subsection{Versionamento}
   \subsubsection{Codice di versione}
   La storia di una risorsa deve sempre poter essere ricostrubibile in quanto nel suo arco di vita essa subisce
   svariate modifiche. A tal scopo è fondamentale l'introduzione di un sistema di identificazione della 
   versione: 
   \begin{center}
       \textbf{[X].[Y].[Z]}
   \end{center}
   Dove:
   \begin{itemize}
       \item \textbf{X} indica la versione stabile, quella approvata dal responsabile del documento
       \item \textbf{Y} indica la versione controllata, quella controllata dal verificatore del documento
       \item \textbf{Z} indica la versione modificata, quella sottoposta a modifica da parte di un redattore
   \end{itemize}


 \subsection{Sistemi software utilizzati}
 Per la gestione delle differenti versioni è stato deciso di utilizzare il sistema
 di versionamento distribuito Git; utilizzando il servizio GitHub per gestire la repository.\\
 Per la suddivisione e la gestione dei lavori da svolgere il nostro gruppo ha deciso di appoggiarsi
 al servizio web Trello.\\

 \subsection{Struttura repository}
  \subsubsection{Repository utilizzata}
  Il repository in cui tutti i documeti vengono caricati è pubblico e si trova al seguente indirizzo: 
  \url{https://github.com/merlunipd/login-warrior}.

  \subsubsection{Organizzazione del repository}
  L'organizzazione del repository è riassunta di seguito:
  \begin{itemize}
      \item Ramo principale \textbf{Main} in cui è presente la documentazione, o le parti dei documenti, approvate da almeno un verificatore e dal responsabile
      \item Rami secondari utilizzati per redarre i vari paragrafi della documentazione che, una volta unititi al ramo principale attraverso l'attività di merging, vengono eliminati per evitare di ostruire il repository.
  \end{itemize}
  Nel ramo Main i documenti si trovano all'interno della cartella "documenti", all'interno della quale possiamo trovare:
  \begin{itemize}
      \item La cartella "candidatura" che contiene i documenti inerenti alla candidatura al capitolato C5
      \item La cartella "interni" che contiene i documenti utili al gruppo
      \item La cartella "esterni" contenente i documenti da fornire ai committenti e al proponente.
  \end{itemize}
  Sempre nel Main possiamo trovare il file ".gitignore" necessario ad evitare il tracciamento dei file \LaTeX{} non necessari ai fini del progetto.


\section{Gestione della Qualità}

\section{Verifica}
\subsection{Scopo}
La verifica ha come obiettivo il controllo che un prodotto sia corretto e completo.

\subsection{Descrizione}
Nella verifica si prende in input ogni processo e lo si controlla, in modo da identificare dubbi o incorrettezze.
Si applica la verifica su un processo quando:
\begin{itemize}
  \item Si raggiunge un livello di maturità adeguato e sufficiente.
  \item A seguire di un cambiamento di stato.
\end{itemize}

\subsection{Indicazioni}
Per assicurare il corretto svolgimento della verifica si devono rispettare i seguenti punti:
\begin{itemize}
  \item Seguire procedure definite.
  \item Seguire criteri validi e affidabili.
  \item Ogni prodotto deve passare attraverso fasi successive, che verranno quindi verificate.
\end{itemize}
Una volta terminata la fase di verifica, rispettando i punti appena citati si può quindi proseguire alla fase di \textbf{Validazione}.

\subsection{Verifica della documentazione}
Il processo di verifica per quanto riguarda la documentazione può essere riassunto in queste attività:
\begin{itemize}
  \item Controllo dell'ortografia e della sintassi.
  \item Controllo della stesura del documento.
  \item Controllo della pertinenza e della correttezza dei contenuti del documento.
\end{itemize}

\subsubsection{Attività di analisi statica}
Questa tipologia di analisi risulta molto utile per la verifica di un documento. Si divide in:
\begin{itemize}
  \item \textbf{Walkthrough}: Tecnica dove il verificatore va alla ricerca di eventuali errori attraverso una lettura ad ampio spettro.
  \item \textbf{Inspection}: Tecnica dove il verificatore va alla ricerca di specifici errori attraverso letture mirate.
\end{itemize}
Inizialmente l'attività di Walkthrough sarà prevalente rispetto a Inspection ma
con l'avanzamento dell'attività di progetto e quindi con il ripetersi del processo di Verifica sulla documentazione, si riuscirà a sviluppare una lista degli errori più comuni detta \textbf{Lista di Controllo} consentendo l'utilizzo più massivo della tecnica \textbf{Inspection} che risulta essere più efficiente.

\section{Validazione}
