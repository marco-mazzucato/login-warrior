\chapter{Processi Primari}

\section{Fornitura}
\section{Sviluppo}

\subsection{Descrizione}
  Lo scopo del processo di sviluppo è quello di raggruppare compiti e attività relative alla codifica di un prodotto software, applicandole durante il suo intero ciclo di vita. \\
  Al fine di produrre un software che rispetti le aspettative del proponente è necessario: 
  \begin{itemize}
    \item Determinare i vincoli tecnologici;
    \item Determinare gli obiettivi di sviluppo e design;
    \item Realizzare un prodotto software che superi tutti i test di verifica e di validazione.
  \end{itemize}
  \subsection{Attività}

  Il processo di sviluppo prevede l'esecuzione delle seguenti attività:
  \begin{itemize}
    \item \textbf{Analisi dei Requisiti}
    \item \textbf{Progettazione}
    \item \textbf{Codifica}
  \end{itemize}
  \subsection{Analisi dei Requisiti}
    L'\textit{Analisi dei Requisiti} è quell'attività che precede lo sviluppo e ha la funzione di:
    \begin{itemize}
      \item Definire lo scopo del prodotto da realizzare;
      \item Definire gli attori del sistema;
      \item Fissare le funzionalità del prodotto;
      \item Fornire una visone più chiara del problema ai progettisti;
      \item Fornire un riferimento ai verificatori per l'attività di controllo dei test;
      \item Fornire una stima della mole di lavoro.
    \end{itemize}
    \subsubsection{Scopo}
    Lo scopo dell'attività è quello di fornire in un documento tutti i requisiti individuati.\\
    Al fine di individuarli è necessario:
    \begin{itemize}
      \item Leggere e comprendere la specifica del capitolato;
      \item Mantenere un confronto costante con il proponente.
    \end{itemize}
    \subsubsection{Casi d'uso}
    Definiscono uno scenario in cui uno o più attori interagiscono con il sistema. Sono identificati nel modo seguente:
    \begin{center}
      \textbf{UC[Numero caso d'uso][Sottocaso]-[Titolo caso d'uso]}\\
    \end{center}
    \subsubsection{Struttura dei requisiti}
      Il codice identificativo di ciascun requisito è di seguito riportato:
      \begin{center}
        \textbf{R[Importanza][Tipologia][Codice]}\\
      \end{center}
      con:
      \begin{itemize}
        \item Importanza: 
        \begin{itemize}
          \item \textbf{1} requisito obbligatorio;
          \item \textbf{2} requisito desiderabile ma non obbligatorio;
          \item \textbf{3} requisito opzionale.
        \end{itemize}
  
        \item Tipologia: 
        \begin{itemize}
          \item \textbf{V} Vincolo;
          \item \textbf{P} Prestazionale;
          \item \textbf{Q} Qualitativo;
          \item \textbf{F} Funzionale.
        \end{itemize}
  
        \item Codice:\\
        Identificativo univico del requisito.
      \end{itemize}

  \subsection{Progettazione}
  \subsubsection{Scopo}
  Lo scopo della progettazione è quello di definire una possibile soluzione ai requisiti evidenziati dall'analisi.
  \subsubsection{Descrizione}
  La progettazione è formata da due parti:
  \begin{itemize}
    \item \textbf{Progettazione logica:} motiva le tecnologie, i framework e le librerie usate per la realizzaione di un prodotto, dimostrandone l'adeguatezza nel PoC.
    \\Contiene:
    \begin{itemize}
      \item I framework e le tecnologie utilizzate;
      \item Il Proof of Concept;
      \item I diagrammi UML.
    \end{itemize}
    \item \textbf{Progettazione di dettaglio:} illustra la base architetturale del prodotto coerentemente a ciò che è previsto nella Progettazione logica.
    \\Contiene:
    \begin{itemize}
      \item Diagrammi delle classi;
      \item Tracciamento delle classi;
      \item Test di unità per ogni componente.
    \end{itemize}
  \end{itemize}
  
  \subsection{Codifica}
  \subsubsection{Scopo}
  Lo scopo della codifica è quello di implementare le specifiche individuate in un prodotto utilizzabile.
  \subsubsection{Commenti}
  Nel caso sorga la necessità di scrivere qualche commento al codice è preferibile che esso sia chiaro e conciso.
  \subsubsection{Nomi dei files}
  I nomi dei files devono:
  \begin{itemize}
    \item Essere univoci;
    \item Esplicitare il contenuto dei files stessi.
  \end{itemize}
  
