\chapter{Processi Organizzativi}

\section{Gestione di Processo}

\subsection{Coordinamento}

\subsubsection{Comunicazioni}
Le \textbf{comunicazioni interne} sono principalmente tra studenti del gruppo e sono conversazioni tra pari. Dipendentemente dal canale di comunicazione scelto, per favorire la produttività, bisognerà adottare un registro adeguato e seguire le convenzioni decise (consultare la sezione \nameref{infrastruttura} per più informazioni sugli strumenti di organizzazione utilizzati):
\begin{itemize}
  \item \textbf{Trello}: comunicazioni formali, brevi e concise di natura principalmente organizzativa;
  \item \textbf{GitHub}: comunicazioni formali, brevi e concise di natura principalmente tecnica;
  \item \textbf{Discord}: in base alla categoria del canale
  \begin{itemize}
    \item \textbf{Risorse}: comunicazioni formali di varia lunghezza che permettono al team di essere sempre aggiornato;
    \item \textbf{Testuali}: comunicazioni di vario tipo: formali (ad esempio nel canale \texttt{generale}), informali (ad esempio nel canale \texttt{off-topic});
    \item \textbf{Vocali}: comunicazioni di vario tipo in accordo con la situazione (ad esempio: semi-formali per riunioni interne; informali per "chiacchiere tra colleghi").
  \end{itemize}
  \item \textbf{WhatsApp}: in base alla chat utilizzata
  \begin{itemize}
    \item \textbf{Gruppo}: comunicazioni semi-formali, brevi e strettamente inerenti al progetto;
    \item \textbf{Individuale}: comunicazioni informali.
  \end{itemize}
  \item \textbf{Google Calendar}: comunicazioni formali, brevi e concise in forma di informazioni complementari ad un evento di calendario.
\end{itemize}

Le \textbf{comunicazioni esterne} sono principalmente con committente e proponente. Queste hanno generalmente un valore molto più elevato e vanno adeguatamente preparate. Gli strumenti principali coinvolti sono: \textbf{Zoom} e \textbf{Google Mail} (attraverso l'indirizzo di gruppo \texttt{merlunipd@gmail.com}). Il registro è sempre formale e dipendentemente dal contesto verrà utilizzato un vocabolario tecnico adeguato. È compito del \textit{Responsabile di Progetto} gestire le comunicazioni esterne.

\subsubsection{Riunioni}
Le \textbf{riunioni interne} sono principalmente tra membri del gruppo. Per come è attualmente organizzato il gruppo vengono svolte a necessità, tipicamente non passano più di 10 giorni tra una e l'altra. È compito del \textit{Responsabile di Progetto} organizzare tali incontri in data e orario che permetta all'intero gruppo di partecipare. In caso di problemi, si prova ad accordarsi in modo da poter partecipare tutti. Nell'eventualità ciò non fosse possibile, chi non riesce a partecipare potrà avere accesso al verbale dell'incontro e se necessario il Responsabile si occuperà di fornire tutti gli aggiornamenti fondamentali. I \textit{Verbali} vengono redatti con modalità round-robin dai membri del gruppo.

Per mantenere efficiente il tempo di lavoro sincrono delle riunioni interne, si adottano i seguenti accorgimenti:
\begin{itemize}
  \item Scaletta: ciascuna riunione segue una scaletta standard (riportata sotto);
  \item Moderazione: lo scriba del verbale avrà anche il ruolo di moderatore, cercando di mantenere la durata della riunione pari o inferiore a quanto stabilito ma riuscendo a discutere di tutti gli argomenti pianificati in modo completo;
  \item Preparazione: ciascun membro del gruppo si impegna a partecipare in modo attivo e produttivo agli incontri e questo richiede un minimo di preparazione pregressa. Il tempo di lavoro sincrono è prezioso e va utilizzato per confrontarsi su argomenti di cui si ha un certo grado di competenza (quantomeno avere un'idea di "cosa so" e "cosa non so").
\end{itemize}

La scaletta di una tipica riunione interna è la seguente:
\begin{itemize}
  \item Controllo task boards: verifica e possibile discussione di tasks (di Trello, GitHub) che necessitano di essere discusse tutti insieme;
  \item Attività specifiche di riunione: attività pianificate ad hoc per una particolare riunione;
  \item Tempo di slack: utilizzato per discutere di argomenti non previsti. Se l'attività di pianificazione è efficace e non ci sono imprevisti, questo spazio non sarà utilizzato.
\end{itemize}
È compito del \textit{Responsabile di Progetto} riportare, espandere con le attività ad hoc e tenere aggiornata la scaletta di ciascun incontro sul calendario Google Calendar condiviso.

Le \textbf{riunioni esterne} sono principalmente quelle con committente e proponente. Generalmente saranno richieste dal gruppo in caso di necessità di opinioni più esperte su questioni tecniche o di way of working e per controllare il corretto progresso del progetto didattico. Il tempo persona di un esperto ha un altissimo valore, quindi le riunioni esterne dovranno avere un adeguato livello di preparazione e avere una durata contenuta.

\subsubsection{Reperibilità}
Ogni membro del gruppo è libero di organizzare il proprio tempo di lavoro individuale asincrono come preferisce, in accordo con altri impegni accademici, personali e quanto dichiarato nel preventivo di periodo.

Come compromesso tra efficacia di comunicazione asincrona e protezione delle tempo personale, i membri del gruppo si impegnano ad essere reperibili per questioni relative al progetto didattico col seguente orario: \textbf{dal lunedì al venerdì dalle 9 alle 17}. Qualsiasi estensione o riduzione di orari o giorni di reperibilità può essere concordata con i membri del gruppo. Questo orario di reperibilità non va considerato né utilizzato come tempo di lavoro attivo ma rappresenta un limite di disponibilità da offrire ai compagni di progetto in caso di necessità.

\subsection{Pianificazione}

\subsubsection{Ruoli di progetto}
I ruoli che ciascun membro dovrà ricoprire durante il corso del progetto, per un tempo congruo rispetto a quanto preventivato, sono:
\begin{itemize}
  \item \textbf{Responsabile di Progetto}: ha il ruolo di guidare il progetto a livello macroscopico e gestire lo svolgimento dei processi, in particolare:
  \begin{itemize}
    \item Essere sempre aggiornato sullo stato di progresso del progetto;
    \item Gestire la pianificazione delle attività di ciascuna milestone, definendole insieme a chi segue il documento \textit{Piano di Progetto} e predisponendo le relative task su GitHub;
    \item Gestire le attività di organizzazione, aggiungendo le task necessarie sulla board di Trello;
    \item Approvare qualsiasi task sia stata completata e verificata, sia di processi primari e di supporto (su GitHub, approvando le pull request), sia di processi organizzativi (su Trello);
    \item Gestire il calendario condiviso;
    \item Gestire le comunicazioni esterne.
  \end{itemize}
  \item \textbf{Amministratore}: ha il ruolo di garantire l'efficacia e l'efficienza dei processi, in particolare:
  \begin{itemize}
    \item Redigere i documenti: \textit{Norme di Progetto}, \textit{Piano di Progetto}, \textit{Piano di Qualifica};
    \item Gestire l'infrastruttura e gli strumenti utilizzati;
    \item Automatizzare i processi;
    \item Individuare punti di miglioramento nei processi.
  \end{itemize}
  \item \textbf{Analista}: ha un ruolo fondamentale nelle fasi iniziali del progetto, comprendendo a fondo le necessità del proponente ed individuando i requisiti fondamentali che la fase di progettazione dovrà soddisfare. Nello specifico si occupa di:
  \begin{itemize}
    \item Redigere il documento: \textit{Analisi dei Requisiti};
    \item Studiare il dominio applicativo relativo alle richieste del proponente;
    \item Scomporre le esigenze del proponente in elementi atomici da poter risolvere singolarmente.
  \end{itemize}
  \item \textbf{Progettista}: ha il compito di modellare i requisiti individuati nella fase di analisi e ricomporli in un'architettura che possa soddisfarli. Nello specifico si occupa di:
  \begin{itemize}
    \item Produrre un'architettura che soddisfi i requisiti richiesti;
    \item Approfondire conoscenze tecniche e ricercare strumenti tecnologici utili nell'ambito di applicazione;
    \item Produrre una soluzione con un alto livello di manutenibilità, seguendo le "best practices" note;
    \item Produrre l'architettura di un sistema con il minimo numero di dipendenze possibili (basso grado di accoppiamento, alto grado di coesione tra componenti).
  \end{itemize}
  \item \textbf{Programmatore}: ha il ruolo di implementare l'architettura prodotta nella fase di progettazione, in particolare:
  \begin{itemize}
    \item Scrivere codice che soddisfa le specifiche di progettazione;
    \item Conoscere ed applicare le "best practices" note riguardanti la scrittura di codice;
    \item Permettere un alto grado di manutenibilità del codice, versionando e documentando;
    \item Scrivere i test relativi al codice prodotto;
    \item Redigere il documento: \textit{manuale utente}.
  \end{itemize}
  \item \textbf{Verificatore}: si occupa di controllare che le attività svolte rispettino il livello di qualità atteso. Per questioni ovvie, il Verificatore non può effettuare il controllo di un'attività svolta da lui stesso.
\end{itemize}

Il ruolo di Responsabile verrà assegnato a turno a tutti i membri del gruppo per una durata di circa \textbf{due settimane}. In questo modo sarà possibile per i membri del gruppo (possibilmente tutti) ricoprire questo ruolo in diversi stadi di avanzamento del progetto, beneficiando nei secondi mandati dell'esperienza maturata.

Gli altri ruoli di progetto non saranno assegnati a periodo ma ad attività. Quest'organizzazione più fluida ha lo scopo di aumentare la flessibilità del gruppo.

\subsubsection{Gestione delle task}
In generale il gruppo è organizzato per predisporre tutte le attività di progresso per ciascuna milestone, in modo che i membri del gruppo possano autonomamente auto-assegnarsi le task (in assenza di accordi preesistenti) e lavorare il più possibile in parallelo e in asincrono. Il compito di definizione, predisposizione e configurazione delle task viene svolto subito dopo la pianificazione delle milestone dal \textit{Responsabile di Progetto} seguendo quanto previsto dal \textit{Piano di Progetto}.

Per task di processi primari o di supporto (come la documentazione), si utilizza GitHub. Il loro ciclo di vita ha il seguente schema:
\begin{itemize}
  \item Creazione: la task definita viene aperta sotto forma di "Issue" su GitHub;
  \item Auto-assegnazione: un membro del gruppo può autonomamente prendere in carico una task non assegnata;
  \item Completamento: la task viene completata, tipicamente su un branch distinto;
  \item Pull request: viene fatta una pull request del branch contenente l'attività completata, collegando la richiesta alla relativa issue;
  \item Verifica: un Verificatore effettua il controllo della qualità;
  \item Accettazione: quando il Verificatore conferma la validità della task, il \textit{Responsabile di Progetto} esegue un rapido controllo e se tutto è in ordine approva la pull request, chiude la issue relativa e cancella il branch dell'attività.
\end{itemize}

Il ciclo di vita delle task di un processo organizzativo viene gestito da Trello ed è descritto nella sezione \nameref{infrastruttura}.

In generale le dimensioni di una task sono variabili rispetto al processo considerato, in pratica sono attività non triviali ma sufficientemente contenute da poter essere svolte da un individuo in un tempo ragionevole. Ad esempio:
\begin{itemize}
  \item (processo primario) Codifica di una classe;
  \item (processo di supporto) Stesura di una sezione di un documento;
  \item (processo organizzativo) Creazione e configurazione delle attività di una milestone.
\end{itemize}

Sia le task primarie e di supporto (gestite su Github) sia quelle organizzative (gestite su Trello), hanno la possibilità di essere discusse individualmente utilizzando la sezione "Commenti" della rispettiva piattaforma. Questa funzionalità risulta fondamentale nel momento in cui l'autore e il Verificatore di una task hanno bisogno di instaurare una discussione. Questo approccio favorisce anche la tracciabilità delle attività svolte e dei relativi problemi incontrati.



\section{Infrastruttura}
\label{infrastruttura}
Fanno parte dell'infrastruttura organizzativa tutti gli strumenti che permettono al gruppo di attuare in modo efficace ed efficiente i processi organizzativi. In particolare tali strumenti permettono la \textbf{comunicazione}, il \textbf{coordinamento} e la \textbf{pianificazione}.

\subsection{Strumenti}
\subsubsection{Trello}
Principale strumento di project management utilizzato come \textbf{Kanban}.

La \textbf{board principale} è divisa nelle seguenti liste:
\begin{itemize}
  \item \textbf{Backlog}: contiene task da fare in un momento non specificato. Questa lista viene esaminata nel momento di definizione di una milestone come integrazione per assicurarsi di non dimenticare obiettivi fondamentali;
  \item \textbf{Todo}: contiene task da fare il prima possibile. Se una task non è assegnata a nessuno, qualsiasi membro del gruppo può autonomamente prenderla in carico e portare avanti il suo ciclo di vita. L'unico che può inserire nuove task in questa lista è il \textit{Responsabile di Progetto};
  \item \textbf{Doing}: contiene task in corso. Da questo punto del ciclo di vita, qualsiasi task ha almeno un membro associato a essa, non possono esserci task "orfane". Questo permette di coordinarsi per essere sicuri che i membri del gruppo non stiano lavorando asincronamente a una stessa task;
  \item \textbf{Verify}: contiene task completate che necessitano di verifica. Qualsiasi membro del gruppo può autonomamente diventare Verificatore di una task aggiungendosi a essa, con la precondizione che non sia la stessa persona che l'ha svolta. In caso di problemi può nascere una conversazione nella sezione "Commenti" tra l'esecutore e il Verificatore, con possibile ritorno della task nella lista "Doing". Quando il Verificatore conferma l'idoneità, può segnalare la sua approvazione con il corrispettivo tag "Approved". Sarà poi compito del \textit{Responsabile di Progetto} accettare la fine della task e spostarla in "Done";
  \item \textbf{Done}: contiene task completate, verificate e accettate. Utile principalmente come traccia storica dei compiti svolti (chi, cosa, quando, problemi).
\end{itemize}

Ciascuna \textbf{task} è costituita da:
\begin{itemize}
  \item \textbf{Titolo} sintetico nella seguente forma generale: \textit{[categoria] - [breve titolo significativo]} (e.g. \texttt{documentazione - verbale meeting interno 23/11/2021});
  \item \textbf{Descrizione} opzionale e breve per dettagli importanti;
  \item \textbf{Membro/i} del gruppo coinvolti nella task;
  \item \textbf{Tag} per segnalare particolarità della task. Le etichette principali definite sono:
  \begin{itemize}
    \item \texttt{Importante}: segnala una task con alto livello di priorità;
    \item \texttt{Team}: segnala una task che necessita del coinvolgimento dell'intero team. Rimane comunque che ci deve essere un membro incaricato Responsabile per quella determinata task;
    \item \texttt{Approved}: segnala una task in "Verify" che è stata verificata e può essere spostata (dal \textit{Responsabile di Progetto}) in "Done".
  \end{itemize}
  \item \textbf{Commenti} per discutere (soprattutto asincronamente) sulla specifica task.
\end{itemize}
Le task inserite nel sistema non vengono \underline{mai cancellate} ma solo spostate tra le liste presenti. Solo in casi eccezionali le task possono essere archiviate ma anche in questo caso di loro e della loro storia resta traccia.

Per navigare più facilmente nella bacheca è possibile impostare dei \textbf{filtri}, ad esempio per membro o tag.

\subsubsection{GitHub}
È il principale servizio di hosting della repository di gruppo e di controllo della versione distribuito.

Generalmente il workflow adottato dal gruppo è il GitHub Flow, che sinteticamente segue il seguente schema:
\begin{itemize}
  \item Riallineamento della repository locale con quella remota;
  \item Creazione di un branch locale su cui effettuare le modifiche;
  \item Push del branch locale verso repository remota;
  \item Creazione di una pull request;
  \item Verifica e successivo merge del branch con le modifiche;
  \item Eliminazione del branch utilizzato dalla repository remota.
\end{itemize}
Per i dettagli consultare la documentazione ufficiale:
\begin{itemize}
  \item \url{https://docs.github.com/en/get-started/quickstart/github-flow}
\end{itemize}

GitHub offre un sistema di "Issue" e "Milestone" per pianificare e coordinare le attività da svolgere. Tipicamente sarà compito del \textit{Responsabile di progetto} predisporre le issue relative a una particolare milestone in modo che il resto del team possa avere più autonomia, potendo autoassegnarsi issue ancora aperte. Idealmente le issue saranno collegate con una o più pull request.

\subsubsection{Discord}
Principale strumento di \textbf{comunicazione interna sincrona} e \textbf{asincrona}. Vengono utilizzati 3 categorie di canali:
\begin{itemize}
  \item \textbf{Canali Risorse}: condivisione di risorse, creazione di sondaggi per effettuare decisione problematiche, integrazione con strumenti esterni per permettere notifiche (e.g. GitHub);
  \item \textbf{Canali Testuali}: comunicazioni testuali sincrone e asincrone tra i membri del gruppo;
  \item \textbf{Canali Vocali}: comunicazioni vocali tra i membri del gruppo, con possibilità di condividere lo schermo se necessario.
\end{itemize}
Ciascuna categoria può contenere un numero variabile di canali, a seconda delle necessità del periodo.

\subsubsection{WhatsApp}
Principale strumento di \textbf{comunicazione interna testuale asincrona}. Viene utilizzato in due modalità:
\begin{itemize}
  \item \textbf{Gruppo}: chat condivisa utilizzata, con parsimonia, per comunicazioni rivolte a tutti i membri;
  \item \textbf{Individuale}: ogni membro del gruppo può essere contattato singolarmente.
\end{itemize}

\subsubsection{Google Calendar}
Calendario condiviso del gruppo utilizzato per comunicare e ricordare:
\begin{itemize}
  \item \textbf{Meeting Interni}: di cui saranno specificati:
  \begin{itemize}
    \item Orario di inizio;
    \item Moderatore;
    \item Scriba (redazione verbale);
    \item Argomenti: specifiche idee da trattare durante una riunione del gruppo.
  \end{itemize}
  \item \textbf{Meeting Esterni}: con proponente o committente;
  \item \textbf{Scadenze Interne};
  \item \textbf{Scadenze Esterne};
  \item Qualsiasi altra attività o evento che può essere collocato in un tempo specifico.
\end{itemize}
Mantenere il calendario aggiornato è compito del \textit{Responsabile di Progetto}.

\subsubsection{Google Drive}
Strumento utilizzato come:
\begin{itemize}
  \item \textbf{Directory condivisa} dai membri del gruppo per documenti temporanei o non ufficiali;
  \item Accesso alla \textbf{suite Google}: Docs, Sheets, Slides.
\end{itemize}

\subsubsection{Zoom}
Strumento di videochiamata utilizzato principalmente per la comunicazione esterna con committente e proponente.

\subsubsection{Google Mail}
Utilizzo dell'indirizzo e-mail condiviso \texttt{merlunipd@gmail.com} per le \textbf{comunicazione esterna} come gruppo con i proponenti e il committente.

\section{Miglioramento}
\todo{Da redigere}
\section{Formazione}
Per garantire un andamento organizzato, simultaneo e alla pari, senza lasciare indietro nessuno, e che quindi favorisca un miglior lavoro
asincrono, ogni componente del gruppo, in caso di lacune, dovrà studiarsi in autonomia gli strumenti e le tecnologie utilizzate per documentare e sviluppare
il progetto, oppure condividere eventuali conoscenze con gli altri membri per velocizzare questo processo di formazione.
\newline
Di seguito sono riportati gli strumenti e le tecnologie utilizzate, con i principali riferimenti usati dal gruppo:
\begin{itemize}
    \item LaTeX: \url{https://www.overleaf.com/learn};
    \item Git: \url{https://docs.github.com/en/get-started/using-git/about-git};
    \item GitHub: \url{https://docs.github.com};
    \item GitHub Flow: \url{https://docs.github.com/en/get-started/quickstart/github-flow}.
\end{itemize}
