\section{Diario della riunione}
\begin{itemize}
  \item \textbf{Analisi dei Requisiti}: 
  \begin{itemize}
    \item Convenzioni UML adottate: Titolo del "confine di sistema" dipendente dal livello di astrazione su cui si lavora:
    \begin{itemize}
      \item Sistema ("Login Warrior") o sezione del sistema (e.g. "Inizializzazione del sistema") per UC di "alto livello" (UCx);
      \item "UCx" di riferimento se stiamo trattando sottocasi o UC di basso livello (UCx.x);
      \item "UCx.x" di riferimento se stiamo trattando sotto-sottocasi (UCx.x.x) o UC con livello di astrazione coerente;
      \item se necessario continuare seguendo il pattern deducibile da questa lista però avere UCx.x.x.x forse è un po' eccessivo. 
    \end{itemize}
    \item Revisione e correzione di UC problematici;
    \item Deciso che gli errori verranno trattati come UC separati ma collegati con gli UC che li sollevano. Inoltre sembra che gli errori devono essere UC di "alto livello" (UCx) e non possono essere sottocasi (UCx.x) (sarebbe da chiedere conferma al docente);
    \item Deciso di raccogliere una prima bozza di requisiti sul file Google Drive. Ciascun membro si occuperà almeno di segnalare quelli relativi ai UC su cui ha lavorato;
    \item Cenni sulla classificazione dei requisiti (funzionali, di qualità, di vincolo, prestazionali).
  \end{itemize}
  \item \textbf{Piano di Progetto}: In vista della posticipazione dell'incontro con il proponente, si è deciso di mantenere invariata la data di fine milestone (\textit{10/01/2022}) ma aggiustare il numero di ore produttive. L'incontro con il proponente e la conclusione dell'\textbf{Analisi dei Requisiti} saranno spostate nella successiva milestone;
  \item Incontro con il proponente: proposto per il giorno \textbf{11/01/2022};
  \item GitHub: chiarimento dell'utilizzo di etichette per segnalare la corretta verifica delle pull request.  
\end{itemize}
