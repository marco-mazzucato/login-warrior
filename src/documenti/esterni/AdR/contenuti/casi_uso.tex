\chapter{Casi d'uso}

\section{UC1}

\section{UC2}

\section{UC3 - Selezione dimensioni}
 \begin{itemize}
     \item \textbf{Attore primario:} Utente.
     \item \textbf{Precondizioni:} L'utente ha selezionato la tipologia di grafico con cui visualizzare i dati [UC2].
     \item \textbf{Postcondizioni:} Le dimensioni vengono aggiornate nel sistema e inizia la personalizzazione della visualizzazione [UC4].
     \item \textbf{Scenario principale:}
     \begin{enumerate}
         \item Viene mosrata all'utente la dimensione di default e ulteriori dimiensioni da cui scegliere;
         \item Ogni dimensione presenta un appoosito riquadro da selezionare nel caso la si voglia utilizzare;
         \item L'utente seleziona la dimensione che più ritiene utile.
     \end{enumerate}
 \end{itemize}

\section{UC4}

\section{UC5}

\section{UC6}

\section{UC7}