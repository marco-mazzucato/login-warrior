\chapter{Qualità di processo}
Per garantire un prodotto stabile e di qualità entro i costi e tempi stabiliti nel \textit{Piano d Progetto}, il gruppo \textit{MERL} ha deciso di adottare lo standard \textit{SPICE}. Questo standard garantisce la qualità di tutti i processi attraverso una definizione chiara degli obiettivi e di soglie minime prestabilite da rispettare.
Per quanto riguarda il miglioramento continuo nella qualità dei processi si è deciso di utilizzare il \textit{Ciclo di Deming}, questo garantisce una qualità tesa al miglioramento continuo dei processi e all'utilizzo ottimale delle risorse, e prevede una costante integrazione tra ricerca, progettazione, verifica e produzione.
\section{Obiettivi}
\begin{center}
  \begin{tabular}{|p{2.5cm}|p{8cm}|p{1.5cm}|} \hline
    \textbf{Obiettivo} & \textbf{Descrizione} & \textbf{Metrica}  \\ \hline
    Budget & Evitare differenze eccessive rispetto al costo preventivato & MPC1 \newline MPC2 \newline MPC3     \\ \hline
    Formazione & Ciascun componente del gruppo deve possedere un livello adeguato di preparazione, per cercare di evitare ritardi nella produzione   & \    \\ \hline
    Calendario & Assicurare una pianificazione adatta ai compiti da svolgere, con conseguente massimizzazione dell'efficienza della produzione  & MPC4       \\ \hline
  \end{tabular}
\end{center}
\section{Metriche}
\begin{center}
  \begin{tabular}{|p{2cm}|p{4cm}|p{4cm}|p{3.5cm}|} \hline
    \textbf{Metrica} & \textbf{Nome} & \textbf{Valore Accettabile} & \textbf{Valore Ottimale}    \\ \hline
      MPC1 & Budget at Completion & Errore del +/- 5\% rispetto al preventivo & Corrispondente al preventivo        \\ \hline
      MPC2 & Budget Variance    & 0\%  & +/- 10\% rispetto al preventivo     \\ \hline
      MPC3 & Actual Cost    & Minore del budget totale  & Corrispondente al preventivo     \\ \hline
      MPC4 & Schedule Variance    & 7 giorni di ritardo/anticipo   & 0 giorni di ritardo/anticipo     \\ \hline
  \end{tabular}
\end{center}
