\chapter{Mitigazione dei Rischi}

\section{Rischi legati alle persone}

\begin{table}[H]
    \centering
    \begin{tabular}{|p{2cm}|p{10cm}|}
    \hline
    \multicolumn{2}{|c|}{\textbf{Disponibilità}} \\ \hline
    \multicolumn{1}{|l|}{\textit{Descrizione}} & Il gruppo si è imbattuto in fasi temporali in cui i membri sono stati più o meno attivi in base agli altri impegni universitari. \\ \hline
    \multicolumn{1}{|l|}{\textit{Mitigazione}} & In base agli errori di organizzazione evidenziati dalle differenze tra preventivo e consuntivo, il gruppo ha iniziato a dare maggior importanza alla fase di preventivazione delle ore in modo da essere il più possibile coerenti con le effettive disponibilità dei membri. \\ \hline
    \end{tabular}
\end{table}

\begin{table}[H]
    \centering
    \begin{tabular}{|p{2cm}|p{10cm}|}
    \hline
    \multicolumn{2}{|c|}{\textbf{Problemi interpersonali}} \\ \hline
    \multicolumn{1}{|l|}{\textit{Descrizione}} & All'inizio i membri del gruppo non si conoscevano tra di loro e questo avrebbe potuto provocare incomprensioni interne. \\ \hline
    \multicolumn{1}{|l|}{\textit{Mitigazione}} & Fin da subito le riunioni sono state eseguite in modo che tutti potessero avere il giusto spazio di intervento favorendo la partecipazione attiva di tutti i membri. È evidente l'impegno da parte di tutti nell'ascoltare e accogliere le proposte altrui. \\ \hline
    \end{tabular}
\end{table}

\begin{table}[H]
    \centering
    \begin{tabular}{|p{2cm}|p{10cm}|}
    \hline
    \multicolumn{2}{|c|}{\textbf{Mancanza di esperienza personale}} \\ \hline
    \multicolumn{1}{|l|}{\textit{Descrizione}} & La poca esperienza dei membri all'interno di un progetto vasto e complesso ha portato ad alcune situazioni di difficoltà. \\ \hline
    \multicolumn{1}{|l|}{\textit{Mitigazione}} & All'interno del gruppo è presente uno spirito di collaborazione che ha permesso aiuti reciproci in situazioni di difficoltà. In questo modo dove un membro si è trovato in difficoltà c'è sempre stato un altro membro pronto a supportarlo al fine di risolvere le difficoltà insieme. \\ \hline
    \end{tabular}
\end{table}



\section{Rischi legati all'organizzazione}

\begin{table}[H]
    \centering
    \begin{tabular}{|p{2cm}|p{10cm}|}
    \hline
    \multicolumn{2}{|c|}{\textbf{Scarsa pianificazione}} \\ \hline
    \multicolumn{1}{|l|}{\textit{Descrizione}} & La pianificazione di un progetto di queste dimensioni risulta difficile, ancor di più con scarsa esperienza in merito. Si è visto infatti che più di una volta il consuntivo è risultato lontano dal preventivo per quanto riguarda le tempistiche. \\ \hline
    \multicolumn{1}{|l|}{\textit{Mitigazione}} & Il gruppo ha capito l'importanza di un preventivo fatto bene e per questo con il passare del tempo è stata data maggior importanza alla fase di preventivazione. Per facilitare questo processo è stato deciso di pianificare milestone brevi con obiettivi chiari e tangibili. \\ \hline
    \end{tabular}
\end{table}



\section{Rischi legati alle tecnologie e agli strumenti}

\begin{table}[H]
    \centering
    \begin{tabular}{|p{2cm}|p{10cm}|}
    \hline
    \multicolumn{2}{|c|}{\textbf{Strumenti sconosciuti}} \\ \hline
    \multicolumn{1}{|l|}{\textit{Descrizione}} & La buona riuscita di un progetto prevede l'utilizzo di strumenti non sempre conosciuti. \\ \hline
    \multicolumn{1}{|l|}{\textit{Mitigazione}} & Il gruppo ha deciso insieme quali strumenti risultavano utili man mano che il progetto avanzava e in seguito a queste decisioni è risultato fondamentale che ogni membro ritagliasse una porzione del proprio tempo per imparare ad utilizzare gli strumenti non conosciuti. \\ \hline
    \end{tabular}
\end{table}

\begin{table}[H]
    \centering
    \begin{tabular}{|p{2cm}|p{10cm}|}
    \hline
    \multicolumn{2}{|c|}{\textbf{Tecnologie sconosciute}} \\ \hline
    \multicolumn{1}{|l|}{\textit{Descrizione}} & La codifica del software richiede chiaramente la conoscenza di tecnologie specifiche, non sempre conosciute. \\ \hline
    \multicolumn{1}{|l|}{\textit{Mitigazione}} & Il gruppo ha discusso insieme su quali fossero le tecnologie più adatte alla codifica mettendo in evidenza pro e contro delle soluzioni. Per completare questo processo ogni membro ha utilizzato parte del suo tempo nello studio individuale delle tecnologie non conosciute. \\ \hline
    \end{tabular}
\end{table}



\section{Rischi legati ai requisiti}

\begin{table}[H]
    \centering
    \begin{tabular}{|p{2cm}|p{10cm}|}
    \hline
    \multicolumn{2}{|c|}{\textbf{Analisi dei requisiti incompleta}} \\ \hline
    \multicolumn{1}{|l|}{\textit{Descrizione}} & L'\textit{Analisi dei Requisiti} è parte fondamentale per la realizzazione del prodotto finale, per questo è necessario che sia completa ed esaustiva. \\ \hline
    \multicolumn{1}{|l|}{\textit{Mitigazione}} & L'\textit{Analisi dei Requisiti} è stata realizzata approfondendo il più possibile i casi d'uso e i requisiti anche grazie ad un confronto diretto con il proponente del progetto in modo che fosse ben chiaro come dovrà essere il prodotto finale. Questo ha permesso al gruppo di effettuare, fino a questo momento, un'analisi soddisfacente. \\ \hline
    \end{tabular}
\end{table}
